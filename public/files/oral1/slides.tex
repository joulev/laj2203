% arara: xelatex
% https://gist.github.com/andrejbauer/ac361549ac2186be0cdb
\documentclass[12pt,aspectratio=169]{beamer}
\usepackage{xeCJK}
\setsansfont{Noto Sans CJK JP}
\setCJKmainfont{Noto Sans CJK JP}
\usepackage{ruby}
\usepackage{tikz}
\usepackage{booktabs}
\usepackage{pgfpages}
\usetheme{CambridgeUS}
\colorlet{myred}{darkred!80!black}
\def\emph#1{{\color{myred}#1}}
\setbeameroption{show notes on second screen=right}
\setbeamertemplate{note page}{\pagecolor{yellow!10}\insertnote}
\setbeamertemplate{navigation symbols}{}
\setbeamertemplate{itemize item}{{\color{myred}$\blacktriangleright$}}
\setbeamertemplate{page number in head/foot}{}
\let\OLDitemize\itemize
\renewcommand\itemize{\OLDitemize\addtolength{\itemsep}{1ex}}
\def\arr{→}
\def\nextnotice{\emph{[next slide]}}
\AtBeginNote{\setlength\parskip{1em}\setlength\itemsep{1em}\vspace{5mm}\par}
% only working for notes. for in-slide furigana, you gotta set it manually
\def\rubysep{-1ex}
\renewcommand\baselinestretch{1.5}
\def\rubysize{0.55}
\title{オーラル1}
\author{Vu Van Dung\\ヴ・ヴァン・ズン}
\date{2021年9月30日}
\begin{document}
\begin{frame}
  \titlepage
  \note{こんばんは、みんなさん。先週は、高校生と\ruby{交}{こう}\ruby{流}{りゅう}で、
  色々なことが分かるようになったから、今日は、一番気がしたことについて、話します。それは、
  日本語四にも、高校生にも、アニメが好きな人が多いということです。}
\end{frame}
\begin{frame}{アニメが好きな人が多い}
  \begin{itemize}
    \item \ruby{期}{き}\ruby{待}{たい}: アニメが好きな人の\ruby{割}{わり}\ruby{合}{あい}は5\ruby{割}{わり}くらい
    \item<2-> 本当: 8割くらい{\scriptsize\color{gray}(私のグループ)}\arr 思ったよりも多かった
    \item<3-> アニメについて話した時間がかなり長かった
    \begin{itemize}
      \item 高校生は熱心に話した
      \item \vspace{-1ex}自分のコレクションを見せてくれた
    \end{itemize}
  \end{itemize}
  \note[item]{日本語のモジュールなので、きっとアニメが好きな人が多いと思って、\ruby{割}{わり}\ruby{合}{あい}
  は五\ruby{割}{わり}くらいだと思っていました。\nextnotice}
  \note<2->[item]{しかし、本当は、私が思ったよりも、その\ruby{割}{わり}\ruby{合}{あい}の方が高いのを分かるようになりました。例えば、私のグル
  ープは八\ruby{割}{わり}くらいでした。\nextnotice}
  \note<3->[item]{それで、プロジェクトでアニメの話の時間はかなり長かったですよ。一番好きなアニメとか、どんなアニメをよく見ている
  とか、私たちは\ruby{熱}{ねっ}\ruby{心}{しん}にチャットしました。高校生は自分のコレクションを見せてくれたこともあります。}
\end{frame}
\begin{frame}{高校生のメッセージ}
  \begin{figure}[h]
    \centering
    \begin{tikzpicture}[every node/.style={inner xsep=5mm}]
      \node[text width=.8\linewidth] (quote1) {アニメが好きな人が多くてうれしかった。};
      \node[anchor=north west,font=\scriptsize\color{gray}]
        (quote1-src) at (quote1.south west) {高校生からのメッセージ(グループ14)};
      \draw[myred,line width=1mm] (quote1-src.south west) -- (quote1.north west);
      % can't believe I have to do this workaround
      \def\rubysep{-1ex}
      \renewcommand\baselinestretch{1.5}
      \node[text width=.8\linewidth,anchor=north west] (quote2) at ([yshift=-8mm]quote1-src.south west)
        {日本のアニメが好きで日本語の勉強を始めた[…]と聞いて、なぜか\ruby{嬉}{うれ}しくなりました。};
      \node[anchor=north west,font=\scriptsize\color{gray}]
        (quote2-src) at (quote2.south west) {高校生からのメッセージ(グループ17)};
      \draw[myred,line width=1mm] (quote2-src.south west) -- (quote2.north west);
    \end{tikzpicture}
  \end{figure}
  \note{それに、私たちがアニメが好きで、高校生たちはうれしいと\ruby{書}{か}きました。スライドで\ruby{書}{か}いてあるも
  のは高校生からのメッセージです。
  
  「アニメが好きな人が多くてうれしかった」とか、「なぜかうれしくなりました」とか、高校生は\ruby{書}{か}きま
  した。}
\end{frame}
\begin{frame}{理由を考えてみる(NUS大学生)}
  \begin{itemize}
    \item アニメは多くの人の日本語の勉強の理由
    
    \arr 日本語のモジュールにアニメが好きな人はたくさんいる{\scriptsize\color{gray}(私も)}
  \end{itemize}
  \begin{figure}[h]
    \centering
    \begin{tikzpicture}[every node/.style={inner xsep=5mm}]
      \node[text width=.8\linewidth] (quote1) {[\dots] everyone was a weeb and
      super passionate about Japanese culture [\dots]};
      \node[anchor=north west,font=\scriptsize\color{gray}]
        (quote1-src) at (quote1.south west) {Ember Shenさん \; --- \; LAJ2203のNUSModsレビュー};
      \draw[myred,line width=1mm] (quote1-src.south west) -- (quote1.north west);
    \end{tikzpicture}
  \end{figure}
  \note{日本語四の学生にとっては、なぜアニメが好きな人はそんな多いでしょう。それは、必ず、
  日本語の勉強の理由がアニメである人は多いですからね。日本で住むつもりの人と、日本の会社に働く
  つもりの人がいるかもしれませんが、大体は私のように、\ruby{漫}{まん}\ruby{画}{が}とかアニメ
  とか、日本のポップカルチャーが好きだから日本語の勉強を始めたでしょうね。だから、アニメのファンが
  ここでたくさんいる理由だと思います。}
\end{frame}
\begin{frame}{理由を考えてみる(高校生)}
  \begin{itemize}
    \item 日本人だから、外国人も日本のものが好きでうれしいはず
    \item<2-> それでも、そんな多くて、\ruby{不}{ふ}\ruby{思}{し}\ruby{議}{ぎ}だ
  \end{itemize}
  \note[item]{それで、日本人の高校生はどうでしょう。日本人だから、外国人が日本のものが好きだということは、
  必ずうれしいですね。もし私の国に興味がある人に会ったら、私もうれしくなります。\nextnotice}
  \note<2->[item]{それでも、八\ruby{割}{わり}なんて、その割合はまだ本当に高いです。なぜでしょうか。それはとても面白いトピック
  だと思います。日本の\ruby{青}{せい}\ruby{年}{ねん}のカルチャーのものは、もう一つ、
  プロジェクトで勉強になりました。}
\end{frame}
\begin{frame}
  \begin{figure}[h]
    \centering
    \begin{tikzpicture}
      \node[font=\Huge\color{myred}] at (0,0) {終};
      \node[align=center] at (0,-2) {ありがとうございました!};
    \end{tikzpicture}
  \end{figure}
  \note{では、ここで私のプレゼンテーションが終わります。今まで聞いてくれて、ありがとうございまし
  た。今はQ\&Aのセクションなので、ご質問がありますか。}
\end{frame}
\end{document}
